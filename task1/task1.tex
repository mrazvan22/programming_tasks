\documentclass[11pt,a4paper]{report}
\usepackage{amsmath,amssymb,calc,ifthen}
\usepackage{float}
%\usepackage{cancel}
\usepackage[table,usenames,dvipsnames]{xcolor} % for coloured cells in tables
\usepackage{tikz}
% Allows us to click on links and references!
\usepackage{hyperref}
\usepackage{url}
\hypersetup{
colorlinks,
citecolor=black,
filecolor=black,
linkcolor=black,
urlcolor=black
}
% Nice package for plotting graphs
% See excellent guide:
% http://www.tug.org/TUGboat/tb31-1/tb97wright-pgfplots.pdf
\usetikzlibrary{plotmarks,shapes}
\usepackage{amsmath,graphicx}
\usepackage{epstopdf}
\usepackage{caption}
\usepackage{subcaption}
\usepackage{graphicx}
% highlight - useful for TODOs and similar
\usepackage{color}
\newcommand{\hilight}[1]{\colorbox{yellow}{#1}}
\newcommand\ci{\perp\!\!\!\perp} % perpendicular sign
\newcommand*\rfrac[2]{{}^{#1}\!/_{#2}} % diagonal fraction
\newcommand\SLASH{\char`\\}
\usepackage{listings}
% margin size
\usepackage[margin=1.1in]{geometry}
\usepackage{pdfpages}
\usepackage{enumitem} % for nested enumerate numbers 1 1.1 1.1.1

\usepackage{titlesec} % reduce spacing after subsections

% \titlespacing\subsection{0pt}{4pt plus 4pt minus 2pt}{4pt plus 2pt minus 2pt}

\definecolor{mygreen}{rgb}{0,0.3,0}
\definecolor{mygray}{rgb}{0.5,0.5,0.5}
\definecolor{mymauve}{rgb}{0.58,0,0.82}

\usepackage{listings}

\lstset{ %
  backgroundcolor=\color{white},   % choose the background color; you must add \usepackage{color} or \usepackage{xcolor}; should come as last argument
  basicstyle=\footnotesize,        % the size of the fonts that are used for the code
  breakatwhitespace=false,         % sets if automatic breaks should only happen at whitespace
  breaklines=true,                 % sets automatic line breaking
  captionpos=b,                    % sets the caption-position to bottom
  commentstyle=\color{mygreen},    % comment style
  deletekeywords={...},            % if you want to delete keywords from the given language
  escapeinside={\%*}{*)},          % if you want to add LaTeX within your code
  extendedchars=true,              % lets you use non-ASCII characters; for 8-bits encodings only, does not work with UTF-8
%   frame=single,	                   % adds a frame around the code
  keepspaces=true,                 % keeps spaces in text, useful for keeping indentation of code (possibly needs columns=flexible)
  keywordstyle=\color{blue},       % keyword style
  language=Python,                 % the language of the code
  morekeywords={*,...},           % if you want to add more keywords to the set
  numbers=left,                    % where to put the line-numbers; possible values are (none, left, right)
  numbersep=5pt,                   % how far the line-numbers are from the code
  numberstyle=\tiny\color{mygray}, % the style that is used for the line-numbers
  rulecolor=\color{black},         % if not set, the frame-color may be changed on line-breaks within not-black text (e.g. comments (green here))
  showspaces=false,                % show spaces everywhere adding particular underscores; it overrides 'showstringspaces'
  showstringspaces=false,          % underline spaces within strings only
  showtabs=false,                  % show tabs within strings adding particular underscores
  stepnumber=2,                    % the step between two line-numbers. If it's 1, each line will be numbered
  stringstyle=\color{mymauve},     % string literal style
  tabsize=2,	                   % sets default tabsize to 2 spaces
  title=\lstname                   % show the filename of files included with \lstinputlisting; also try caption instead of title
}

\begin{document}
\belowdisplayskip=12pt plus 3pt minus 9pt
\belowdisplayshortskip=7pt plus 3pt minus 4pt

\subsection*{Aims}
\begin{itemize}
 \item practice writing simple functions in python that perform numerical computation and string operations
 \item learn how to test python functions using \emph{nosetools}
\end{itemize}


\subsection*{Problem 1}

\begin{itemize}
 \item Write a function $\textbf{quad}(a, b, c, x)$ that takes as input three floats $a$, $b$, $c$, a value $x$ and returns the value of the quadratic function $f(x) = ax^2 +bx +c$. For example, $quad(1,2,0,1) = 5$ and $quad(1,0,0,1) = 1$. 
 \item Write a function $\textbf{quadIsZero}(a, b, c, x)$ that takes similar arguments as $\textbf{quad}$, calls $\textbf{quad}$ and returns True if the quadratic expression evaluates to zero, otherwise False (return type is boolean). For example, $quadIsZero(1,2,0,1) = False$ and $quad(1,0,-1,1) = True$. 
 \item Write a function $\textbf{quadSolver}(a, b, c, x)$ that takes similar arguments as $\textbf{quad}$ and returns the two roots of a quadratic equation with coefficients $a$,$b$,$c$. The roots value can be calculated using the following formula:
 $$ x = \frac{-b \pm \sqrt{b^2 - 4ac}}{2a}$$
 There are two main cases:
 \begin{itemize}
  \item if $\sqrt{b^2 - 4ac} \ge 0 $ then the roots are real, in which case return them as a tuple $(x_1,x_2)$ where $x_1 < x_2$. 
  \item if $\sqrt{b^2 - 4ac} < 0 $ then the roots are imaginary, in which case return NaN instead of the tuple.
 \end{itemize}
\end{itemize}

\subsection*{Problem 2}

\begin{itemize}
  \item write a function $\textbf{toUpperCase(s)}$ which takes a lowercase string $s$ and converts the letters at the beginning of every word to uppercase. For example: 
  \begin{lstlisting}
  toUpperCase('a very simple example') = 'A Very Simple Example'
  \end{lstlisting}   
  Perform this in two ways:
  \begin{itemize}
   \item Perform a for loop ever every character:
     \begin{lstlisting}
for char in s:
  # check if letter is at the beginning of the word. This requires checking a boolean variable isAtBegWord at every loop
    # if letter is at beginning of word, then transform it to uppercase. 
    
  # if the current character is a space ' ', set the boolean variable isAtBegWord to True, else False.
  \end{lstlisting}  
  \item Use the \textbf{str.split} function with delimiter ' ', which splits the string into words, then update the first letter of each word and assemble the words back together using \textbf{str.join}(wordList). Check on google the documentation of these two functions. Call this function $\textbf{toUpperCase2}$
  
  \end{itemize}


\end{itemize}


\end{document}





















