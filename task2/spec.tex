\documentclass[11pt,a4paper]{report}
\usepackage{amsmath,amssymb,calc,ifthen}
\usepackage{float}
%\usepackage{cancel}
\usepackage[table,usenames,dvipsnames]{xcolor} % for coloured cells in tables
\usepackage{tikz}
% Allows us to click on links and references!
\usepackage{hyperref}
\usepackage{url}
\hypersetup{
colorlinks,
citecolor=black,
filecolor=black,
linkcolor=black,
urlcolor=black
}
% Nice package for plotting graphs
% See excellent guide:
% http://www.tug.org/TUGboat/tb31-1/tb97wright-pgfplots.pdf
\usetikzlibrary{plotmarks,shapes}
\usepackage{amsmath,graphicx}
\usepackage{epstopdf}
\usepackage{caption}
\usepackage{subcaption}
\usepackage{graphicx}
% highlight - useful for TODOs and similar
\usepackage{color}
\newcommand{\hilight}[1]{\colorbox{yellow}{#1}}
\newcommand\ci{\perp\!\!\!\perp} % perpendicular sign
\newcommand*\rfrac[2]{{}^{#1}\!/_{#2}} % diagonal fraction
\newcommand\SLASH{\char`\\}
\usepackage{listings}
% margin size
\usepackage[margin=1.1in]{geometry}
\usepackage{pdfpages}
\usepackage{enumitem} % for nested enumerate numbers 1 1.1 1.1.1

\usepackage{titlesec} % reduce spacing after subsections

% \titlespacing\subsection{0pt}{4pt plus 4pt minus 2pt}{4pt plus 2pt minus 2pt}

\definecolor{mygreen}{rgb}{0,0.3,0}
\definecolor{mygray}{rgb}{0.5,0.5,0.5}
\definecolor{mymauve}{rgb}{0.58,0,0.82}

\usepackage{listings}

\lstset{ %
  backgroundcolor=\color{white},   % choose the background color; you must add \usepackage{color} or \usepackage{xcolor}; should come as last argument
  basicstyle=\footnotesize,        % the size of the fonts that are used for the code
  breakatwhitespace=false,         % sets if automatic breaks should only happen at whitespace
  breaklines=true,                 % sets automatic line breaking
%   captionpos=b,                    % sets the caption-position to bottom
  commentstyle=\color{mygreen},    % comment style
  deletekeywords={...},            % if you want to delete keywords from the given language
  escapeinside={\%*}{*)},          % if you want to add LaTeX within your code
  extendedchars=true,              % lets you use non-ASCII characters; for 8-bits encodings only, does not work with UTF-8
%   frame=single,	                   % adds a frame around the code
  keepspaces=true,                 % keeps spaces in text, useful for keeping indentation of code (possibly needs columns=flexible)
  keywordstyle=\color{blue},       % keyword style
  language=Python,                 % the language of the code
  morekeywords={*,...},           % if you want to add more keywords to the set
  numbers=none,                    % where to put the line-numbers; possible values are (none, left, right)
  numbersep=5pt,                   % how far the line-numbers are from the code
  numberstyle=\tiny\color{mygray}, % the style that is used for the line-numbers
  rulecolor=\color{black},         % if not set, the frame-color may be changed on line-breaks within not-black text (e.g. comments (green here))
  showspaces=false,                % show spaces everywhere adding particular underscores; it overrides 'showstringspaces'
  showstringspaces=false,          % underline spaces within strings only
  showtabs=false,                  % show tabs within strings adding particular underscores
  stepnumber=2,                    % the step between two line-numbers. If it's 1, each line will be numbered
  stringstyle=\color{mymauve},     % string literal style
  tabsize=2,	                   % sets default tabsize to 2 spaces
  title=\lstname,                   % show the filename of files included with \lstinputlisting; also try caption instead of title
  aboveskip=-3pt,
  belowskip=5pt
}

\begin{document}
\belowdisplayskip=12pt plus 3pt minus 9pt
\belowdisplayshortskip=7pt plus 3pt minus 4pt

\subsection*{Aims}
\begin{itemize}
 \item practice writing loops and conditionals
 \item learn how to read from a file
 \item learn how to work with matrices
\end{itemize}

\subsection*{Problem 1 - Addresses}

The aim here is to practice reading and writing files, as well as using dictionaries and tuples. Write your solutions in a file \textbf{sol.py}.

You are given an input file 'addresses.txt' which contains details (names, age, address, job) about several persons. Have a look at the file first and then write the following functions:
\begin{enumerate}
 \item Write a function \textbf{loadFile(filePath)} which takes a string variable filePath, loads the file contents in a dictionary called persDetails and returns the dictionary. In general, a dictonary is a collection of (key, value) pairs. In your case, the key will be a string representing the person's name and the value will be a 3-element tuple (age, address, job). At the end of the script (outside the function), run the following code:
  \begin{lstlisting}
  persDetails = loadFile('addresses.txt')
  print(persDetails)
  \end{lstlisting}  
  This should show (not necessary in this order as dictionaries are not ordered):
  \begin{lstlisting}
  {'Radu': (24, 'Str_Corvin_nr_7', 'programator'), 'Maria': (40, 'Str_Corvin_nr_7', 'economist'), 'Enescu': (30, 'Str_Muzicienilor_nr_1', 'pianist'), 'Ana': (25, 'Str_Maslinilor_nr_3', 'lingvist'), 'George': (18, 'Str_Grozavesti_nr_7', 'student'), 'Ionel': (20, 'Str_Bratianu_nr_4', 'consultant')}
  \end{lstlisting}
    
 \item Write a function \textbf{getAge(persDetails, name)} which takes the previously loaded dictionary persDetails and the name of a person and returns the corresponding age of that person. For example, getAge(persDetails, 'Ionel') = 20 
 \item Write a similar function \textbf{getAddress(persDetails, name)} which works like getAge but returns the corresponding address of that person. For example, getAddress(persDetails, 'Ionel') = 'Str\_Bratianu\_nr\_4'
 \item Write a function \textbf{getNamesFromAge(persDetails, age)} which takes the dictionary persDetails and an age (integer) and returns a list of the names of persons that have this age. For example, getNamesFromAge(persDetails, 24) = ['Radu', 'Ana']
 \item Write a function \textbf{addToDict(persDetails, name, age, address, job)} which takes a dictionary persDetails, a new name, age, address and job and adds them to the dictionary. You might find the function \textbf{dict.keys()} useful. The function should return the updated dictionary. Run the following code to check if the result worked:
   \begin{lstlisting}
persDetails = addToDict(persDetails, 'Adriana', 25, 'Str_Maniu_nr_3', 'medic')
# should show entry with Adriana
print(persDetails)
   \end{lstlisting}
 \item  Write a function \textbf{removeFromDict(persDetails, name)} which takes a dictionary persDetals and a name and removes the entry with the corresponding name from the dictionary. Do not use del or pop. Do this by constructing a new dictionary newDict with all the entries apart from the removed one and return it. You can test your code as follows:
    \begin{lstlisting}
persDetails = removeFromDict(persDetails, 'Enescu')
# Should not show Enescu entry
print(persDetails)
   \end{lstlisting}
 \item Write a function \textbf{writeToFile(persDetails, filePath)} which takes the dictionary and saves it to a file called filePath. Save one dictionary entry on each line. On each line, show the name, age, address and job, separated by a whitespace. Run the function call writeToFile(persDetails, 'new\_addresses.txt') and check the file outputs. 
 
 \item You can use this code for calling all functions:
 \begin{lstlisting}
persDetails = loadFile('addresses.txt')
print(persDetails)

ageIonel = getAge(persDetails, 'Ionel')
print('ageIonel ', ageIonel)
addressIonel = getAddress(persDetails, 'Ionel')
print('addressIonel ', addressIonel)

names24 = getNamesFromAge(persDetails, 24)
print('names24', names24)
persDetails = addToDict(persDetails, 'Adriana', 25, 'Str_Maniu_nr_3', 'medic')
# should show entry with Adriana
print(persDetails)

persDetails = removeFromDict(persDetails, 'Enescu')
# Show not show Enescu entry
print(persDetails)

writeToFile(persDetails, 'new_addresses.txt')

\end{lstlisting}

  
\end{enumerate}

\section*{Problem 2 - Matrices}

The aim here is to practice using matrices in Python using numpy. You need to have a working version of numpy to proceed with this problem. We can define a 1-dimensional numpy vector in python as follows:

\begin{lstlisting}
 import numpy as np # place this at the beginning of the file
 
 m = np.zeros(10, float) # vector has 10 entries, all elements are initialised to zero
 m[0] = 4 # assign the value of 4 to the first element
 print(m[0])
\end{lstlisting}

We can also convert a list of numbers into a (numpy) array as follows:

\begin{lstlisting}
  n = np.array([1,2,3,7,8])
  print(n) # show the value of n
\end{lstlisting}

For declaring a 2-dimensional numpy matrix and accessing an element we can use the following syntax:

\begin{lstlisting}
m = np.zeros((3,3), float) # create a 3x3 matrix where all elements are initialised to zero, float is the type of date being stored in the matrix
m[0,1] = 7 # assign the value of 7 to element at position (0,1)
print(m)
\end{lstlisting}

The following example creates the matrix $\left[ {\begin{array}{cccc}
   1 & 2 & 6\\     
   3 & 4 & 7\\
   \end{array} } \right]$ and stores it in variable m:
   
\begin{lstlisting}
m = np.zeros((3,2), float)
m[0,1] = 1
m[0,2] = 2
m[0,3] = 6
m[1,0] = 3
m[1,1] = 4
m[1,2] = 7
print(m) # try printing the matrix
\end{lstlisting}


\begin{itemize}
 \item Following the examples above, create two matrices and store them in variables m1 and m2:
   \[ m1= \left[ {\begin{array}{cccc}
   1 & 2 & 6 & 9\\     
   3 & 4 & 7 & 1\\
   \end{array} } \right]
\]
   \[ m2= \left[ {\begin{array}{cccc}
   1 & 2 & 3 & 1\\     
   5 & 6 & 7 & 0\\
   \end{array} } \right]
\]

 \item Write a function \textbf{printMatrix(m)} that takes a 4x2 matrix m and loops through each element of m and prints it on the screen. Do not use any other higher-level functions such as print(). You need to loop through each element m[i,j] (requires two nested for-loops, one until 4 and the other until 2) and print it, along with an extra space character. At the end of each row, don't forget to print a newline character. 
 
 \item Write a function \textbf{addMatrix(m,n)} that takes two numpy matrices m and n and returns m+n. Do not use the high-level operator '+'. Instead, create a new numpy matrix p and loop over every element p[i,j] of the matrix p (requires two nested for-loops) and assign it the correct value. You can assume m,n have size 4x2.
 Test your function using the matrixes m1 and m2 defined above.
 
 \item Write a function \textbf{transposeMatrix(m)} which computes and returns $m^T$, the transpose of matrix m.  The transpose of a matrix m with dimensions AxB is a matrix $m^T$ with dimensions BxA, such that $m^T[i,j] = m[j,i]$ (elements are flipped around the diagonal). You can assume m has size 4x2. For example, for matrix m1 defined above, it's transpose is:
    \[ transposeMatrix(m1)= \left[ {\begin{array}{cccc}
   1 & 3\\     
   2 & 4\\
   6 & 7\\
   9 & 1\\
   \end{array} } \right]
\]
 \item Write a function \textbf{slideMatrix(m)} which returns a new matrix n, where all elements are slided to the right by one position. Moreover, the elements on the last column are placed on the first column. For example:
 
    \[ slideMatrix(m1)= \left[ {\begin{array}{cccc}
   9 & 1 & 2 & 6\\     
   1 & 3 & 4 & 7\\
   \end{array} } \right]
\]

 \item HARDER (you can skip this if too difficult): Given two matrices $M:A \times B$ (i.e. $M$ has $A$ rows and $B$ columns) and $N:B \times C$ we can multiply then them to get $P=MN$, where $P$ is an $A \times C$ matrix defined as 
 
 $$P[i,j] = \sum_{k=0}^{B-1} M[i,k] * N[k,j], \forall i \in [0 \dots (A-1)], j \in [0 \dots (C-1)]$$
 
 For example, if we consider $ m1= \left[ {\begin{array}{cccc}
   1 & 2 & 6 & 9\\     
   3 & 4 & 7 & 1\\
   \end{array} } \right]
$ and $m3 = \left[ {\begin{array}{cccc}
   1 & 3\\     
   0 & 4\\
   -2 & 0\\
   9 & 1\\
   \end{array} } \right]$, then 
   
   $$ matrixMultiply(m1, m3) = \left[ {\begin{array}{cccc}
  70 & 24\\
  -2 & 26\\
   \end{array} } \right]$$
 
 Write a function \textbf{multiplyMatrix(m,n)} which takes as input a matrix 2x4 matrix m, a 4x2 matrix n and returns a new matrix p=m*n, the result of multiplying matrices m and n. 
 
 \item Modify all the functions above to accept matrices of any shape (so far we only assumed 4x2 matrices). If we are given a matrix m, we can find it's shape using function \textbf{m.shape}. We can then use this to create a new matrix n of the same size as 
 
 \begin{lstlisting}
shapeM = m.shape 
n = np.zeros(shapeM, float) # create a matrix n of the same size as m
\end{lstlisting}
 Use this strategy to change your previous functions to accept matrices of any shape. Test all your functions for new matrices of different sizes. 
 
 
\end{itemize}




\end{document}








